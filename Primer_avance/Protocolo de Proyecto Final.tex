\documentclass[10pt,letterpaper]{article}
\usepackage[utf8]{inputenc}
\usepackage[margin=0.9in]{geometry}
\usepackage[spanish]{babel} 
\usepackage{amsmath}
\usepackage{amsfonts}
\usepackage{amssymb}
\usepackage{amsthm}
\usepackage{graphicx}
\usepackage{subfigure}
\usepackage{xcolor}

\usepackage{hyperref}
\usepackage{atbegshi,picture}
\AtBeginShipout{\AtBeginShipoutUpperLeft{%
  %\put(\dimexpr\paperwidth-1cm\relax,-1.5cm){\makebox[0pt][r]{Copyright DTV}}%
  \put(\dimexpr1.5cm\relax,-2cm){\makebox[0pt][l]{\shortstack[l]{Programaci\'on Orientada a Objetos y Eventos \\ Enero - Junio 2021 \\ Dr. Luis Carlos Padierna Garc\'ia}}}
}}

\usepackage{multicol}
\usepackage{enumitem}
\setlist{  
	listparindent=\parindent,
	parsep=0pt,
}

\newcommand{\N}{\mathbb{N}}
\newcommand{\Z}{\mathbb{Z}}
\newcommand{\Q}{\mathbb{Q}}
\newcommand{\R}{\mathbb{R}}
\newcommand{\C}{\mathbb{C}}

\newcommand{\abs}[1]{\left| #1 \right|}

%\makeatletter
%\newcommand\re{\mathop{\operator@font Re}\nolimits}
%\newcommand\im{\mathop{\operator@font Im}\nolimits}
%\newcommand\grad{\mathop{\operator@font grad}\nolimits}
%\newcommand\Fr{\mathop{\operator@font Fr}\nolimits}
%\makeatother

\newcommand{\contradiction}{\Rightarrow\!\Leftarrow}

\theoremstyle{definition}
\newtheorem*{solution}{Solución}
\newtheorem{definition}{Definición}
\newtheorem{theorem}{Teorema}
\newtheorem{proposition}{Proposición}
\newtheorem*{notation}{Notación}

\title{Programando el juego Go}
\author{Oliva Ramírez Adrián Fernando \\ NUA: 424647 \\ \href{mailto:af.olivaramirez@ugto.mx}{af.olivaramirez@ugto.mx}}

\begin{document}

\maketitle

\section{Introducci\'on}

El juego de mesa llamado Go, es el juego m\'as antiguo del que se tiene conocimiento. Es un juego de estrategia de dos jugadores y el objetivo es conquistar y controlar m\'as \'area del tablero que el contrincante. Las reglas del juego son sencillas de entender, pero es dif\'icil jugar bien.

En este proyecto programar\'e el juego de Go, y tal vez implemente un modo ``en l\'inea'' para jugar con otras personas en el internet.

\section{Objetivo}

Usar tkinter con gr\'aficos para programar un tablero de Go jugable. Tambi\'en investigar maneras para que se pueda jugar con otras personas en el internet.

\end{document}
