\documentclass[10pt,letterpaper]{article}
\usepackage[utf8]{inputenc}
\usepackage[margin=0.9in]{geometry}
\usepackage[spanish]{babel} 
\usepackage{amsmath}
\usepackage{amsfonts}
\usepackage{amssymb}
\usepackage{amsthm}
\usepackage{graphicx}
\usepackage{subfigure}
\usepackage{xcolor}

\usepackage{tikz}
\usepackage{tikz-uml}

\usepackage{hyperref}
\usepackage{atbegshi,picture}
\AtBeginShipout{\AtBeginShipoutUpperLeft{%
  %\put(\dimexpr\paperwidth-1cm\relax,-1.5cm){\makebox[0pt][r]{Copyright DTV}}%
  \put(\dimexpr1.5cm\relax,-2cm){\makebox[0pt][l]{\shortstack[l]{Programaci\'on Orientada a Objetos y Eventos \\ Enero - Junio 2021 \\ Dr. Luis Carlos Padierna Garc\'ia}}}
}}

\usepackage{multicol}
\usepackage{enumitem}
\setlist{  
	listparindent=\parindent,
	parsep=0pt,
}

\newcommand{\N}{\mathbb{N}}
\newcommand{\Z}{\mathbb{Z}}
\newcommand{\Q}{\mathbb{Q}}
\newcommand{\R}{\mathbb{R}}
\newcommand{\C}{\mathbb{C}}

\newcommand{\abs}[1]{\left| #1 \right|}

%\makeatletter
%\newcommand\re{\mathop{\operator@font Re}\nolimits}
%\newcommand\im{\mathop{\operator@font Im}\nolimits}
%\newcommand\grad{\mathop{\operator@font grad}\nolimits}
%\newcommand\Fr{\mathop{\operator@font Fr}\nolimits}
%\makeatother

\newcommand{\contradiction}{\Rightarrow\!\Leftarrow}

\theoremstyle{definition}
\newtheorem*{solution}{Solución}
\newtheorem{definition}{Definición}
\newtheorem{theorem}{Teorema}
\newtheorem{proposition}{Proposición}
\newtheorem*{notation}{Notación}

\title{Programando el juego Go}
\author{Oliva Ramírez Adrián Fernando \\ NUA: 424647 \\ \href{mailto:af.olivaramirez@ugto.mx}{af.olivaramirez@ugto.mx}}

\begin{document}

\maketitle

\section{Introducci\'on}

El juego de mesa llamado Go, es el juego m\'as antiguo del que se tiene conocimiento. Es un juego de estrategia de dos jugadores y el objetivo es conquistar y controlar m\'as \'area del tablero que el contrincante. Las reglas del juego son sencillas de entender, pero es dif\'icil jugar bien.

En este proyecto programar\'e el juego de Go, y tal vez implemente un modo ``en l\'inea'' para jugar con otras personas en el internet.

\section{Objetivo}

Usar \verb|pygame| con gr\'aficos para programar un tablero de Go jugable. Tambi\'en investigar maneras para que se pueda jugar con otras personas en el internet.

\section{Justificaci\'on}

El juego de Go tiene pocas reglas y es sencillo de entender. El problema es que existen detalles, como calcular el puntaje final, que no son sencillos de programar. Incluso apenas en 2015 es que la primera m\'aquina, \textit{AlphaGo}, logr\'o vencer a un jugador profesional. Se necesit\'o de inteligencia artificial para poder programar a \textit{AlphaGo}.

Quiero tener un acercamiento al juego y poder entenderlo lo suficiente como para poder programarlo. Aparte si tengo el tiempo suficiente para poder programar el modo ``en l\'inea'', entender\'e de mejor manera el env\'io y recibo de paquetes.

\section{Diagrama de Clases}

Para facilitar el trabajo, usaremos una clase \verb|Cluster| para guardar los grupos de las fichas. Aquella clase podr\'a actualizar dichos grupos y llevar registro de las ``libertades''. Cuando un grupo de fichas se queda sin libertades, el grupo entero es capturado.

Por otra parte, tendremos la clase del \verb|Tablero Go|. Aquel guardar\'a el tablero entero, los clusters del tablero y se encargar\'a de que todas las reglas de Go sean aplicadas en cada movimiento. Heredaremos las clases \verb|Tablero 19|, \verb|Tablero 13| y \verb|Tablero 9|, donde el n\'umero indica el tama\~no del tablero. Esto para poder adaptar la funci\'on \verb|dibujarTablero()| para cada tablero.

Tenemos el siguiente diagrama:

\begin{tikzpicture} 
\umlclass{Cluster}{
    + color : bool \\
    -- fichasDelCluster : set[tuple[int]] \\
    -- puntosDeLibertad : set[tuple[int]]
    }{
    + Cluster(x : int, y : int) \\
    + actualizarCluster() : void \\
    + numLibertades() : int \\
    + fichaEnLibertad(x : int, y : int) : bool \\
    + enemigoVecinos() : list[Cluster] \\
    + calcPerimetro() : list[tuple[int]]
}

\umlclass[x=9]{Tablero Go}{ 
    \# tamano : int \\
    \# tablero : matrix[int] \\
    \# tablero1Antes : matrix[int] \\
    \# tablero2Antes : matrix[int] \\
    \# listaClusters : list[Cluster] \\
    \# clustTablero : matrix[int]
    }{ 
    + Tablero\_Go() \\
    + esEspacioValido(x : int, y : int) : bool \\
    + ponerFicha(x : int, y : int) : void \\
    + calcularPuntaje() : tuple[int] \\
    \umlvirt{+ dibujarTablero() : void}
}

\umldep[geometry=--]{Cluster}{Tablero Go}

\umlclass[y=-6]{Tablero 19}{

    }{
    + Tablero\_19() \\
    + dibujarTablero() : void
}
\umlinherit[geometry=|-|]{Tablero 19}{Tablero Go}

\umlclass[x=5, y=-6]{Tablero 13}{

    }{
    + Tablero\_13() \\
    + dibujarTablero() : void
}
\umlinherit[geometry=|-|]{Tablero 13}{Tablero Go}

\umlclass[x=10, y=-6]{Tablero 9}{

    }{
    + Tablero\_9() \\
    + dibujarTablero() : void
}
\umlinherit[geometry=|-|]{Tablero 9}{Tablero Go}
\end{tikzpicture}

\section{Actividades de Programaci\'on}

\begin{itemize}
    \item Renderizar im\'agenes y usar el mouse como m\'etodo de entrada usando \verb|pygame|.

    \item Programar un tablero de Go jugable con todas las reglas b\'asicas, sin la funci\'on de calcular el puntaje.

    \item Dibujar de manera correcta el tablero de Go y usar el mouse para poner fichas.

    \item Implementar un sistema para calcular el puntaje de los dos jugadores y declarar un ganador. \cite{scoreGo}

    \item (Opcional) Investigar sobre el env\'io y recibo de sockets por el internet. \cite{multijugador}

    \item (Opcional) Implementar el modo ``en l\'inea'' al juego.
\end{itemize}

\begin{thebibliography}{9}
    \bibitem{scoreGo}
    Andrea Carta (2018) \emph{A static method for computing the score of a Go game}, \href{http://www.micini.net/}{http://www.micini.net/}. Visitado el 2021-11-29 en \href{https://www.oipaz.net/Carta.pdf}{https://www.oipaz.net/Carta.pdf}.

    \bibitem{multijugador}
    Tim Ruscica (2019) \emph{Python Online Multiplayer Game Development Tutorial}, \href{https://www.freecodecamp.org/}{www.freecodecamp.org}. Visitado el 2021-11-29 en \href{https://www.youtube.com/watch?v=McoDjOCb2Zo}{https://www.youtube.com/watch?v=McoDjOCb2Zo}.
\end{thebibliography}

\end{document}
